% generated by plotdistinguisher.py
\documentclass[preview]{standalone}
\usepackage{comment}
\usepackage{tugcolors}
\usepackage{twine}
\usepackage[margin=2.65in]{geometry}
%\tikzset{warpfig/.append style={white}}
\def\diffusiondepth{2.65}
\begin{document}
% \begin{figure}[htp!]
\begin{tikzpicture}[twinefig]
\foreach \z in {20, 21, 22, 23} { \fill[twinered] (.25*\z,.75) circle[radius=3pt]; }
   \foreach \z[evaluate=\z as \zf using int(4*\z)] in {0,...,15} {
       \draw[gray] (\z,0) node[above] {\tiny\zf};
       \foreach \zb in {0,...,3} { \draw[gray] (\z+.25*\zb,0) -- +(0,-3pt); }
}
\end{tikzpicture}
% \twineroundwokey{%s
% \markevenbranches{2/1}{markupperpath}{->}
% \markoddbranchbeforexor{1,3}{markupperpath}
% \markoddbranchafterxor{1/0}{markupperpath}{->}
% }
% \twineroundwokey{%s
% \markevenbranches{0/5}{markupperpath}{->}
% \markoddbranchbeforexor{1}{markupperpath}
% \markoddbranchafterxor{}{markupperpath}{->}
% }
\twineroundwokey{%s
\markevenbranches{}{markmidupperpath}{->}
\markoddbranchbeforexor{5}{markmidupperpath}
\markoddbranchafterxor{5/12}{markmidupperpath}{->}
\markevenbranchbeforetee{0,2,4,6,8,10,12,14}{markmidlowerpath}{<-}
\markoddbranch{1/0,3/4,5/12,7/8,9/6,11/2,13/10,15/14}{markmidlowerpath}{<-}
\markevenbranchaftertee{0/5,2/1,4/7,6/3,8/13,10/9,12/15,14/11}{markmidlowerpath}{<-}
\markcommonactivesboxes{}
}
\colorlet{twineyellow}{tugyellow}
\twineroundwokey{%s
\markevenbranches{12/15}{markmidupperpath}{->}
\markoddbranchbeforexor{}{markmidupperpath}
\markoddbranchafterxor{13/10}{markmidupperpath}{->}
\markevenbranchbeforetee{0,2,4,6,8,10,12,14}{markmidlowerpath}{<-}
\markoddbranch{1/0,3/4,5/12,7/8,9/6,11/2,13/10,15/14}{markmidlowerpath}{<-}
\markevenbranchaftertee{2/1,4/7,6/3,8/13,10/9,12/15,14/11}{markmidlowerpath}{<-}
\markcommonactivesboxes{6}
\draw (s6) ++(-.35,.7)  node[twineyellow] {$\Delta_\text{i}$};
\draw (s6) ++(.35,-.7)  node[twineyellow] {$\Delta_\textsc{m}$};
%\draw (b15) ++(.35,.35) node[twineyellow] {$\Delta_\text{i}$};
}
\colorlet{twineyellow}{tugyellow}
\twineroundwokey{%s
\markevenbranches{10/9}{markmidupperpath}{->}
\markoddbranchbeforexor{15}{markmidupperpath}
\markoddbranchafterxor{11/2,15/14}{markmidupperpath}{->}
\markevenbranchbeforetee{0,2,4,6,8,10,12,14}{markmidlowerpath}{<-}
\markoddbranch{1/0,3/4,7/8,9/6,11/2,13/10,15/14}{markmidlowerpath}{<-}
\markevenbranchaftertee{0/5,2/1,4/7,12/15,14/11}{markmidlowerpath}{<-}
\markcommonactivesboxes{5}
\draw (s5) ++(-.35,.7)  node[twineyellow] {$\Delta_\textsc{m}$};
\draw (s5) ++(.35,-.7)  node[twineyellow] {$*$};
}
\colorlet{twineyellow}{tuggreen}
\twineroundwokey{%s
\markevenbranches{2/1,14/11}{markmidupperpath}{->}
\markoddbranchbeforexor{9}{markmidupperpath}
\markoddbranchafterxor{3/4,9/6,15/14}{markmidupperpath}{->}
\markevenbranchbeforetee{0,2,4,6,8,10,14}{markmidlowerpath}{<-}
\markoddbranch{1/0,5/12,7/8,11/2,15/14}{markmidlowerpath}{<-}
\markevenbranchaftertee{2/1,8/13,10/9}{markmidlowerpath}{<-}
\markcommonactivesboxes{7}
\draw (s7) ++(-.35,.7)  node[twineyellow] {$\Delta_\text{i}$};
\draw (s7) ++(.35,-.7)  node[twineyellow] {$\lambda_\textsc{m}$};
}
\twineroundwokey{%s
\markevenbranches{4/7,6/3,14/11}{markmidupperpath}{->}
\markoddbranchbeforexor{1,11}{markmidupperpath}
\markoddbranchafterxor{1/0,5/12,7/8,11/2,15/14}{markmidupperpath}{->}
\markevenbranchbeforetee{0,2,8,12,14}{markmidlowerpath}{<-}
\markoddbranch{1/0,9/6,13/10}{markmidlowerpath}{<-}
\markevenbranchaftertee{2/1,14/11}{markmidlowerpath}{<-}
\markcommonactivesboxes{}
}
\colorlet{twineyellow}{tugyellow}
\twineroundwokey{%s
\markevenbranches{0/5,2/1,8/13,12/15,14/11}{markmidupperpath}{->}
\markoddbranchbeforexor{3,7,11}{markmidupperpath}
\markoddbranchafterxor{1/0,3/4,7/8,9/6,11/2,13/10,15/14}{markmidupperpath}{->}
\markevenbranchbeforetee{0,6,10}{markmidlowerpath}{<-}
\markoddbranch{1/0,11/2}{markmidlowerpath}{<-}
\markevenbranchaftertee{6/3}{markmidlowerpath}{<-}
\markcommonactivesboxes{0}
\draw (s0) ++(-.35,.7)  node[twineyellow] {$*$};
\draw (s0) ++(.35,-.7)  node[twineyellow] {$\lambda_\text{o}$};
}
\colorlet{twineyellow}{tuggreen}
\twineroundwokey{%s
\markevenbranches{0/5,2/1,4/7,6/3,8/13,10/9,14/11}{markmidupperpath}{->}
\markoddbranchbeforexor{1,5,11,13,15}{markmidupperpath}
\markoddbranchafterxor{1/0,3/4,5/12,7/8,9/6,11/2,13/10,15/14}{markmidupperpath}{->}
\markevenbranchbeforetee{0,2}{markmidlowerpath}{<-}
\markoddbranch{3/4}{markmidlowerpath}{<-}
\markevenbranchaftertee{0/5}{markmidlowerpath}{<-}
\markcommonactivesboxes{1}
\draw (s1) ++(-.35,.7)  node[twineyellow] {$\lambda_\textsc{m}$};
\draw (s1) ++(.35,-.7)  node[twineyellow] {$*$};
}
\colorlet{twineyellow}{tuggreen}
\twineroundwokey{%s
\markevenbranches{0/5,2/1,4/7,6/3,8/13,10/9,12/15,14/11}{markmidupperpath}{->}
\markoddbranchbeforexor{1,3,5,7,9,11,13}{markmidupperpath}
\markoddbranchafterxor{1/0,3/4,5/12,7/8,9/6,11/2,13/10,15/14}{markmidupperpath}{->}
\markevenbranchbeforetee{4}{markmidlowerpath}{<-}
\markoddbranch{5/12}{markmidlowerpath}{<-}
\markevenbranchaftertee{}{markmidlowerpath}{<-}
\markcommonactivesboxes{2}
\draw (s2) ++(-.35,.7)  node[twineyellow] {$*$};
\draw (s2) ++(.35,-.7)  node[twineyellow] {$\lambda_\text{o}$};
}
\twineroundwokey{%s
\markevenbranches{0/5,2/1,4/7,6/3,8/13,10/9,12/15,14/11}{markmidupperpath}{->}
\markoddbranchbeforexor{1,3,5,7,9,11,13,15}{markmidupperpath}
\markoddbranchafterxor{1/0,3/4,5/12,7/8,9/6,11/2,13/10,15/14}{markmidupperpath}{->}
\markevenbranchbeforetee{12}{markmidlowerpath}{<-}
\markoddbranch{}{markmidlowerpath}{<-}
\markevenbranchaftertee{12/15}{markmidlowerpath}{<-}
\markcommonactivesboxes{}
}
% \twineroundwokey{%s
% \markuppercrossingdifferences{0,1,2,3,4,5,6,7,8,9,10,11,12,13,14,15}
% \markevenbranchbeforetee{}{marklowerpath}{<-}
% \markoddbranch{15/14}{marklowerpath}{<-}
% \markevenbranchaftertee{14/11}{marklowerpath}{<-}
% }
% \twineroundwokey{%s
% \markevenbranchbeforetee{14}{marklowerpath}{<-}
% \markoddbranch{11/2}{marklowerpath}{<-}
% \markevenbranchaftertee{10/9,14/11}{marklowerpath}{<-}
% \markoutputdiff{2,9,11}
% }
\begin{tikzpicture}[twinefig]
\foreach \z[evaluate=\z as \zf using int(4*\z)] in {0,...,15} {
\draw[gray] (\z,-3pt) node[below] {\tiny\zf};
\foreach \zb in {0,...,3} { \draw[gray] (\z+.25*\zb,0) -- +(0,-3pt); }
}
\foreach \z in {60, 61, 62, 63} { \fill[twineblue] (.25*\z,-.75) circle[radius=3pt]; }
\end{tikzpicture}
\begin{comment}
#######################################################
Summary of the results:
A differential trail for EU:
Rounds	x                 pr     
--------------------------------
0	0a79000000000000  -2                
1	a700000000000000  -2                
2	00000a0000000000  none              
Weight: -4.00
-------------------------------------------------------
Sandwich 9 rounds in the middle with 6 active S-boxes
-------------------------------------------------------
A linear trail for EL:
Rounds	x                 pr     
--------------------------------
0	000000000000000a  -2                
1	00000000000c00a0  -2                
2	00c00000020a0000  none              
Weight: -4.00
#######################################################
differential effect of the upper trail: 2^(-4.00)
squared correlation of the lower trail: 2^(-4.00)
#######################################################

Total correlation = p*r*q^2 = 2^(-4.00) x r x 2^(-4.00)
2^(-17.00) <= Total correlation <= 2^(-11.00)
To compute the accurate value of total probability, r should be evaluated experimentally or using the DLCT framework

Number of attacked rounds: 13
Configuration: ru=2, rm=9, rl=2, rmu=0, rml=0, wu=1, wm=1, wl=1
#######################################################
DLCT: 

8  8  8  8  8  8  8  8  8  8  8  8  8  8  8  8
8  0  0  0  0  0 -4 -4 -4  0  0 -4  0  4  0  4
8  0 -4  0  0  0 -4  0  0 -4  4  4 -4  0  0  0
8  0 -4  0 -4  4  4  0  0  0  0 -4  0  0 -4  0
8 -4 -4  0  0 -4  0  4 -4  4  0  0  0  0  0  0
8  0 -4 -4  4  0  0 -4  4  0 -4  0  0  0  0  0
8 -4  4 -4 -4  0  0  0  0 -4  0  0  0  4  0  0
8  0  4  0  0 -4  0  0  0  0 -4  0 -4  0 -4  4
8 -4  0  0  0  0 -4  0  4  0  0  0  4 -4 -4  0
8  0  0  4  0  4  0  0 -4 -4 -4  0  0 -4  0  0
8 -4  0  4  0  0  4 -4  0  0  0  0 -4  0  0 -4
8  0  0  0 -4  0 -4  0  0  4 -4  0  0  0  4 -4
8  4  0 -4  0  0  0  0 -4  0  0  4  0  0 -4 -4
8  0  0  0  4 -4  0  0  0 -4  0 -4  4  0  0 -4
8  4  0  0 -4 -4  0 -4  0  0  4  0  0 -4  0  0
8  0  0 -4  0  0  0  4  0  0  0 -4 -4 -4  4  0

#######################################################
DDLCT:

128 128 128 128 128 128 128 128 128 128 128 128 128 128 128 128
128   8 -16 -16   0   8  -8  -8 -16  -8 -16  16 -24 -16   0 -32
128 -16 -16   8 -16 -16  16 -24   0  16 -16  -8 -24  -8  -8 -16
128  -8   0 -32  -8 -16 -16   0  16 -16   0  -8 -24 -24   0   8
128   0   0  -8 -32   0  -8   0  -8  -8 -16 -24   0 -16  16 -24
128 -16  -8   8 -16   8  -8 -16 -16 -32   0   0 -32   8 -16   8
128  16  -8 -16  16 -16  -8 -32  -8 -16 -24  -8   0  -8 -16   0
128 -16 -16  -8  -8  16   0   0 -16 -16 -24   0   8 -16 -24  -8
128  -8 -40 -16   0   0   0   0  -8  -8   0 -32  -8   0  -8   0
128 -16   0  -8   0  -8   0   0   8  -8  -8 -16   0 -24 -32 -16
128 -16   8   0  -8 -16 -24   8 -16   8 -32 -16 -16 -16  -8  16
128  -8   0 -16   0 -32 -16   8 -16 -16  -8   0   0   8  -8 -24
128  -8 -16  16  -8 -24  -8 -16  -8 -16  16   0   0 -32  -8 -16
128   8 -16  -8 -24   0 -32 -24 -16   0   8  -8   0   0 -16   0
128 -16  -8 -16 -16 -24   0   0 -32   8   8  -8 -16   0   0  -8
128 -32   8 -16  -8  -8 -16 -24   8 -16 -16 -16   8  16   0 -16
\end{comment}
% \caption{Differential-linear distinguisher for 13 rounds of \texttt{TWINE}.}
% \label{fig:difflin_distinguisher}
% \end{figure}
\end{document}
