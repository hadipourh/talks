%%%%%%%%%%%%%%%%%%%%%%%%%%%%%%%%%%%%%%%%%%%%%%%%%%%%%%%%%%%%%%%%%%%%%%%%%%%%%%%%
%%% TIKZLIBRARY CIPHER %%%%%%%%%%%%%%%%%%%%%%%%%%%%%%%%%%%%%%%%%%%%%%%%%%%%%%%%%
%   Utilities for drawing cryptographic circuits                               %
%   Version: 2021-04-21                                                        %
%   Author: Maria Eichlseder                                                   %
%   \usetikzlibrary{cipher}                                                    %
%%%%%%%%%%%%%%%%%%%%%%%%%%%%%%%%%%%%%%%%%%%%%%%%%%%%%%%%%%%%%%%%%%%%%%%%%%%%%%%%


\makeatletter

%%%%%%%%%%%%%%%%%%%%%%%%%%%%%%%%%%%%%%%%%%%%%%%%%%%%%%%%%%%%%%%%%%%%%%%%%%%%%%%%
%%% SHAPES %%%%%%%%%%%%%%%%%%%%%%%%%%%%%%%%%%%%%%%%%%%%%%%%%%%%%%%%%%%%%%%%%%%%%
%%%%%%%%%%%%%%%%%%%%%%%%%%%%%%%%%%%%%%%%%%%%%%%%%%%%%%%%%%%%%%%%%%%%%%%%%%%%%%%%

%%% UTILS %%%%%%%%%%%%%%%%%%%%%%%%%%%%%%%%%%%%%%%%%%%%%%%%%%%%%%%%%%%%%%%%%%%%%%

% from https://tex.stackexchange.com/questions/14769/add-more-anchors-to-standard-tikz-nodes
\newcommand{\anchorlet}[2]{%
  \global\expandafter
  \let\csname pgf@anchor@\pgf@sm@shape@name @#1\expandafter\endcsname
  \csname pgf@anchor@\pgf@sm@shape@name @#2\endcsname
}

%%% OPERATION SHAPES %%%%%%%%%%%%%%%%%%%%%%%%%%%%%%%%%%%%%%%%%%%%%%%%%%%%%%%%%%%
\pgfdeclareshape{oplus}{%{{{
  \inheritsavedanchors[from=circle]
  \inheritanchorborder[from=circle]
  \foreach \s in {center,mid,base,text, north,south,west,east,
                  mid west,mid east,base west,base east,
                  north west,south west,north east,south east} {
    \inheritanchor[from=circle]{\s}
  }
  %\inheritbackgroundpath[from=circle]
  % From /usr/share/texlive/texmf-dist/tex/generic/pgf/modules/pgfmoduleshapes.code.tex:1148
  \backgroundpath{%
    \pgfutil@tempdima=\radius%
    \pgfmathsetlength{\pgf@xb}{\pgfkeysvalueof{/pgf/outer xsep}}%
    \pgfmathsetlength{\pgf@yb}{\pgfkeysvalueof{/pgf/outer ysep}}%
    \ifdim\pgf@xb<\pgf@yb%
      \advance\pgfutil@tempdima by-\pgf@yb%
    \else%
      \advance\pgfutil@tempdima by-\pgf@xb%
    \fi%
    \pgfpathcircle{\centerpoint}{\pgfutil@tempdima}%
    % north-south
    \centerpoint\advance\pgf@y by\radius  \pgf@xa=\pgf@x \pgf@ya=\pgf@y
    \centerpoint\advance\pgf@y by-\radius \pgf@xb=\pgf@x \pgf@yb=\pgf@y
    \pgfpathmoveto{\pgfpoint{\pgf@xa}{\pgf@ya}}
    \pgfpathlineto{\pgfpoint{\pgf@xb}{\pgf@yb}}
    % east-west
    \centerpoint\advance\pgf@x by\radius  \pgf@xa=\pgf@x \pgf@ya=\pgf@y
    \centerpoint\advance\pgf@x by-\radius \pgf@xb=\pgf@x \pgf@yb=\pgf@y
    \pgfpathmoveto{\pgfpoint{\pgf@xa}{\pgf@ya}}
    \pgfpathlineto{\pgfpoint{\pgf@xb}{\pgf@yb}}
  }
}%}}}

\pgfdeclareshape{ominus}{%{{{
  \inheritsavedanchors[from=circle]
  \inheritanchorborder[from=circle]
  \foreach \s in {center,mid,base,text, north,south,west,east,
                  mid west,mid east,base west,base east,
                  north west,south west,north east,south east} {
    \inheritanchor[from=circle]{\s}
  }
  %\inheritbackgroundpath[from=circle]
  % From /usr/share/texlive/texmf-dist/tex/generic/pgf/modules/pgfmoduleshapes.code.tex:1148
  \backgroundpath{%
    \pgfutil@tempdima=\radius%
    \pgfmathsetlength{\pgf@xb}{\pgfkeysvalueof{/pgf/outer xsep}}%
    \pgfmathsetlength{\pgf@yb}{\pgfkeysvalueof{/pgf/outer ysep}}%
    \ifdim\pgf@xb<\pgf@yb%
      \advance\pgfutil@tempdima by-\pgf@yb%
    \else%
      \advance\pgfutil@tempdima by-\pgf@xb%
    \fi%
    \pgfpathcircle{\centerpoint}{\pgfutil@tempdima}%
    % east-west
    \centerpoint\advance\pgf@x by\radius  \pgf@xa=\pgf@x \pgf@ya=\pgf@y
    \centerpoint\advance\pgf@x by-\radius \pgf@xb=\pgf@x \pgf@yb=\pgf@y
    \pgfpathmoveto{\pgfpoint{\pgf@xa}{\pgf@ya}}
    \pgfpathlineto{\pgfpoint{\pgf@xb}{\pgf@yb}}
  }
}%}}}

\pgfdeclareshape{otimes}{%{{{
  \inheritsavedanchors[from=circle]
  \inheritanchorborder[from=circle]
  \foreach \s in {center,mid,base,text, north,south,west,east,
                  mid west,mid east,base west,base east,
                  north west,south west,north east,south east} {
    \inheritanchor[from=circle]{\s}
  }
  %\inheritbackgroundpath[from=circle]
  % From /usr/share/texlive/texmf-dist/tex/generic/pgf/modules/pgfmoduleshapes.code.tex:1148
  \backgroundpath{%
    \pgfutil@tempdima=\radius%
    \pgfmathsetlength{\pgf@xb}{\pgfkeysvalueof{/pgf/outer xsep}}%
    \pgfmathsetlength{\pgf@yb}{\pgfkeysvalueof{/pgf/outer ysep}}%
    \ifdim\pgf@xb<\pgf@yb%
      \advance\pgfutil@tempdima by-\pgf@yb%
    \else%
      \advance\pgfutil@tempdima by-\pgf@xb%
    \fi%
    \pgfpathcircle{\centerpoint}{\pgfutil@tempdima}%
    % nw-se
    \centerpoint
    \pgf@xc=\radius
    \advance\pgf@x by-0.707107\pgf@xc  \pgf@xa=\pgf@x
    \advance\pgf@y by 0.707107\pgf@xc  \pgf@ya=\pgf@y
    \centerpoint
    \advance\pgf@x by 0.707107\pgf@xc  \pgf@xb=\pgf@x
    \advance\pgf@y by-0.707107\pgf@xc  \pgf@yb=\pgf@y
    \pgfpathmoveto{\pgfpoint{\pgf@xa}{\pgf@ya}}
    \pgfpathlineto{\pgfpoint{\pgf@xb}{\pgf@yb}}
    % sw-ne
    \centerpoint
    \advance\pgf@x by-0.707107\pgf@xc  \pgf@xa=\pgf@x
    \advance\pgf@y by-0.707107\pgf@xc  \pgf@ya=\pgf@y
    \centerpoint
    \advance\pgf@x by 0.707107\pgf@xc  \pgf@xb=\pgf@x
    \advance\pgf@y by 0.707107\pgf@xc  \pgf@yb=\pgf@y
    \pgfpathmoveto{\pgfpoint{\pgf@xa}{\pgf@ya}}
    \pgfpathlineto{\pgfpoint{\pgf@xb}{\pgf@yb}}
  }
}%}}}

\pgfdeclareshape{boxplus}{%{{{
  \inheritsavedanchors[from=rectangle]
  \inheritanchorborder[from=rectangle]
  \foreach \s in {center,mid,base,text, north,south,west,east,
                  mid west,mid east,base west,base east,
                  north west,south west,north east,south east} {
    \inheritanchor[from=rectangle]{\s}
  }
  \backgroundpath{
    % store lower right in xa/ya and upper right in xb/yb
    \southwest \pgf@xa=\pgf@x \pgf@ya=\pgf@y
    \northeast \pgf@xb=\pgf@x \pgf@yb=\pgf@y
    % store center in xc/yc
    \pgf@xc=.5\pgf@xa \advance\pgf@xc by .5\pgf@xb
    \pgf@yc=.5\pgf@ya \advance\pgf@yc by .5\pgf@yb
    % construct main path
    \pgfpathmoveto{\pgfpoint{\pgf@xa}{\pgf@ya}}
    \pgfpathlineto{\pgfpoint{\pgf@xa}{\pgf@yb}}
    \pgfpathlineto{\pgfpoint{\pgf@xb}{\pgf@yb}}
    \pgfpathlineto{\pgfpoint{\pgf@xb}{\pgf@ya}}
    \pgfpathclose
    \pgfpathmoveto{\pgfpoint{\pgf@xc}{\pgf@ya}}
    \pgfpathlineto{\pgfpoint{\pgf@xc}{\pgf@yb}}
    \pgfpathmoveto{\pgfpoint{\pgf@xa}{\pgf@yc}}
    \pgfpathlineto{\pgfpoint{\pgf@xb}{\pgf@yc}}
  }
}%}}}

%%% PRIMITIVE SHAPES %%%%%%%%%%%%%%%%%%%%%%%%%%%%%%%%%%%%%%%%%%%%%%%%%%%%%%%%%%%
\pgfdeclareshape{compressionf}{%{{{
  \inheritsavedanchors[from=rectangle]
  \inheritanchorborder[from=rectangle]

  \foreach \s in {center,base,text,
                  north,south,west,east,
                  north west, south west, north east, south east} {
    \inheritanchor[from=rectangle]{\s}
  }
  \anchor{msg}{\southwest \pgf@xa=\pgf@x
               \northeast \pgf@x=\pgf@xa}
  \anchor{tip}{\southwest \pgf@xa=\pgf@x
               \northeast
               \pgf@x=\pgf@xa
               \advance\pgf@y by \pgf@y}
  \anchor{top}{\southwest \pgf@xa=\pgf@x
               \northeast
               \advance\pgf@x by-.5\pgf@x
               \advance\pgf@x by .5\pgf@xa
               \advance\pgf@y by .5\pgf@y}
  \backgroundpath{
    % store lower right in xa/ya and upper right in xb/yb
    \southwest \pgf@xa=\pgf@x \pgf@ya=\pgf@y
    \northeast \pgf@xb=\pgf@x \pgf@yb=\pgf@y
    \pgf@xc=\pgf@xa
    \pgf@yc=\pgf@yb \advance\pgf@yc by \pgf@yb

    \pgfpathmoveto{\pgfpoint{\pgf@xa}{\pgf@ya}}
    \pgfpathlineto{\pgfpoint{\pgf@xa}{\pgf@yc}}
    \pgfpathlineto{\pgfpoint{\pgf@xb}{\pgf@yb}}
    \pgfpathlineto{\pgfpoint{\pgf@xb}{\pgf@ya}}
    \pgfpathclose
  }
}%}}}

\pgfdeclareshape{compression}{%{{{
  \inheritsavedanchors[from=rectangle]
  \inheritanchorborder[from=rectangle]

  \foreach \s in {center,base,text,
    north,south,west,east,
    north west, south west, north east, south east} {
    \inheritanchor[from=rectangle]{\s}
  }
  \backgroundpath{
    % store lower left in xa/ya and upper right in xb/yb
    \southwest \pgf@xa=\pgf@x \pgf@ya=\pgf@y
    \northeast \pgf@xb=\pgf@x \pgf@yb=\pgf@y
    \pgf@xc=\pgf@xa \advance\pgf@xc by .75\pgf@xa
    \pgf@yc=\pgf@xb \advance\pgf@yc by-.75\pgf@xa

    \pgfpathmoveto{\pgfpoint{\pgf@xa}{\pgf@ya}}
    \pgfpathlineto{\pgfpoint{\pgf@xc}{\pgf@yb}}
    \pgfpathlineto{\pgfpoint{\pgf@yc}{\pgf@yb}}
    \pgfpathlineto{\pgfpoint{\pgf@xb}{\pgf@ya}}
    \pgfpathclose
  }
}%}}}

\pgfdeclareshape{filter}{%{{{
  \inheritsavedanchors[from=rectangle]
  \inheritanchorborder[from=rectangle]

  \foreach \s in {center,base,text,
                  north,south,west,east,
                  north west, south west, north east, south east} {
    \inheritanchor[from=rectangle]{\s}
  }
  \backgroundpath{
    % store lower left in xa/ya and upper right in xb/yb
    \southwest \pgf@xa=\pgf@x \pgf@ya=\pgf@y
    \northeast \pgf@xb=\pgf@x \pgf@yb=\pgf@y
    \pgf@xc=\pgf@xa \advance\pgf@xc by .75\pgf@xa
    \pgf@yc=\pgf@xb \advance\pgf@yc by-.75\pgf@xa

    \pgfpathmoveto{\pgfpoint{\pgf@xa}{\pgf@ya}}
    \pgfpathlineto{\pgfpoint{\pgf@xc}{\pgf@yb}}
    \pgfpathlineto{\pgfpoint{\pgf@yc}{\pgf@yb}}
    \pgfpathlineto{\pgfpoint{\pgf@xb}{\pgf@ya}}
    \pgfpathclose
  }
}%}}}

\pgfdeclareshape{authentication}{%{{{
  \inheritsavedanchors[from=rectangle]
  \inheritanchorborder[from=rectangle]

  \foreach \s in {center,base,text,
                  north,south,west,east,
                  north west, south west, north east, south east} {
    \inheritanchor[from=rectangle]{\s}
  }
  \anchor{tag}{\northeast \pgf@xa=\pgf@x \pgf@ya=\pgf@y
               %\southwest \pgf@x=\pgf@xa}
               \southwest \pgf@x=\pgf@xa \advance\pgf@y by .5\pgf@y \advance\pgf@y by -.5\pgf@ya}
  \anchor{tip}{\northeast \pgf@xa=\pgf@x \pgf@ya=\pgf@y
               \southwest
               \pgf@x=\pgf@xa
               %\advance\pgf@y by \pgf@y}
               \advance\pgf@y by \pgf@y \advance\pgf@y by -\pgf@ya}
  \anchor{bot}{\northeast \pgf@xa=\pgf@x \pgf@ya=\pgf@y
               \southwest
               \advance\pgf@x by-.5\pgf@x
               \advance\pgf@x by .5\pgf@xa
               %\advance\pgf@y by .5\pgf@y}
               \advance\pgf@y by .5\pgf@y \advance\pgf@y by -.5\pgf@ya}
  \backgroundpath{
    % store upper right in xa/ya and lower left in xb/yb
    \northeast \pgf@xa=\pgf@x \pgf@ya=\pgf@y
    \southwest \pgf@xb=\pgf@x \pgf@yb=\pgf@y
    \pgf@xc=\pgf@xa
    %\pgf@yc=\pgf@yb \advance\pgf@yc by \pgf@yb
    \pgf@yc=\pgf@yb \advance\pgf@yc by \pgf@yb \advance\pgf@yc by -\pgf@ya

    \pgfpathmoveto{\pgfpoint{\pgf@xa}{\pgf@ya}}
    \pgfpathlineto{\pgfpoint{\pgf@xa}{\pgf@yc}}
    \pgfpathlineto{\pgfpoint{\pgf@xb}{\pgf@yb}}
    \pgfpathlineto{\pgfpoint{\pgf@xb}{\pgf@ya}}
    \pgfpathclose
  }
}%}}}

\pgfdeclareshape{wide}{%{{{
  \inheritsavedanchors[from=rectangle]
  \inheritanchorborder[from=rectangle]

  \foreach \s in {center,base,text,
                  north,south,west,east,
                  north west, south west, north east, south east} {
    \inheritanchor[from=rectangle]{\s}
  }
  \anchor{wnw}{\northeast \pgf@xa=\pgf@x \pgf@ya=\pgf@y
               \southwest \advance\pgf@y by-.75\pgf@y
                          \advance\pgf@y by .75\pgf@ya}
  \anchor{wsw}{\northeast \pgf@xa=\pgf@x \pgf@ya=\pgf@y
               \southwest \advance\pgf@y by-.25\pgf@y
                          \advance\pgf@y by .25\pgf@ya}
  \anchor{ene}{\southwest \pgf@xa=\pgf@x \pgf@ya=\pgf@y
               \northeast \advance\pgf@y by-.25\pgf@y
                          \advance\pgf@y by .25\pgf@ya}
  \anchor{ese}{\southwest \pgf@xa=\pgf@x \pgf@ya=\pgf@y
               \northeast \advance\pgf@y by-.75\pgf@y
                          \advance\pgf@y by .75\pgf@ya}

  \anchor{nnw}{\southwest \pgf@xa=\pgf@x \pgf@ya=\pgf@y
               \northeast \advance\pgf@x by-.75\pgf@x
                          \advance\pgf@x by .75\pgf@xa}
  \anchor{nne}{\southwest \pgf@xa=\pgf@x \pgf@ya=\pgf@y
               \northeast \advance\pgf@x by-.25\pgf@x
                          \advance\pgf@x by .25\pgf@xa}
  \anchor{ssw}{\northeast \pgf@xa=\pgf@x \pgf@ya=\pgf@y
               \southwest \advance\pgf@x by-.25\pgf@x
                          \advance\pgf@x by .25\pgf@xa}
  \anchor{sse}{\northeast \pgf@xa=\pgf@x \pgf@ya=\pgf@y
               \southwest \advance\pgf@x by-.75\pgf@x
                          \advance\pgf@x by .75\pgf@xa}

  \inheritbackgroundpath[from=rectangle]

  \anchorlet{xouter}{wnw}
  \anchorlet{youter}{ene}
  \anchorlet{xinner}{wsw}
  \anchorlet{yinner}{ese}
}%}}}

% TODO keyed north/west/south/east

%%% OTHER SHAPES %%%%%%%%%%%%%%%%%%%%%%%%%%%%%%%%%%%%%%%%%%%%%%%%%%%%%%%%%%%%%%%
\pgfdeclareshape{document}{%{{{
  % paper with folded corner
  % from PGF manual, "Command for Declare New Shapes"
  \inheritsavedanchors[from=rectangle]
  \inheritanchorborder[from=rectangle]
  %\inheritanchor[from=rectangle]{center}
  %\inheritanchor[from=rectangle]{north}
  %\inheritanchor[from=rectangle]{south}
  %\inheritanchor[from=rectangle]{west}
  %\inheritanchor[from=rectangle]{east}
  \foreach \s in {center,mid,base,text,north,south,west,east,
                  mid west,mid east,base west,base east,
                  north west,south west,north east,south east} {
    \inheritanchor[from=rectangle]{\s}
  }
  \backgroundpath{
    % store lower right in xa/ya and upper right in xb/yb
    \southwest \pgf@xa=\pgf@x \pgf@ya=\pgf@y
    \northeast \pgf@xb=\pgf@x \pgf@yb=\pgf@y
    % compute corner of ‘‘flipped page’’
    \pgf@xc=\pgf@xb \advance\pgf@xc by-5pt % this should be a parameter
    \pgf@yc=\pgf@yb \advance\pgf@yc by-5pt
    % construct main path
    \pgfpathmoveto{\pgfpoint{\pgf@xa}{\pgf@ya}}
    \pgfpathlineto{\pgfpoint{\pgf@xa}{\pgf@yb}}
    \pgfpathlineto{\pgfpoint{\pgf@xc}{\pgf@yb}}
    \pgfpathlineto{\pgfpoint{\pgf@xb}{\pgf@yc}}
    \pgfpathlineto{\pgfpoint{\pgf@xb}{\pgf@ya}}
    \pgfpathclose
    % add little corner
    \pgfpathmoveto{\pgfpoint{\pgf@xc}{\pgf@yb}}
    \pgfpathlineto{\pgfpoint{\pgf@xc}{\pgf@yc}}
    \pgfpathlineto{\pgfpoint{\pgf@xb}{\pgf@yc}}
    \pgfpathlineto{\pgfpoint{\pgf@xc}{\pgf@yc}}
  }
}
\pgfdeclareshape{minidocument}{
  % paper with folded corner
  % from PGF manual, "Command for Declare New Shapes"
  \inheritsavedanchors[from=rectangle]
  \inheritanchorborder[from=rectangle]
  \inheritanchor[from=rectangle]{center}
  \inheritanchor[from=rectangle]{north}
  \inheritanchor[from=rectangle]{south}
  \inheritanchor[from=rectangle]{west}
  \inheritanchor[from=rectangle]{east}
  \backgroundpath{
    % store lower right in xa/ya and upper right in xb/yb
    \southwest \pgf@xa=\pgf@x \pgf@ya=\pgf@y
    \northeast \pgf@xb=\pgf@x \pgf@yb=\pgf@y
    % compute corner of ‘‘flipped page’’
    \pgf@xc=\pgf@xb \advance\pgf@xc by-2pt % this should be a parameter
    \pgf@yc=\pgf@yb \advance\pgf@yc by-2pt
    % construct main path
    \pgfpathmoveto{\pgfpoint{\pgf@xa}{\pgf@ya}}
    \pgfpathlineto{\pgfpoint{\pgf@xa}{\pgf@yb}}
    \pgfpathlineto{\pgfpoint{\pgf@xc}{\pgf@yb}}
    \pgfpathlineto{\pgfpoint{\pgf@xb}{\pgf@yc}}
    \pgfpathlineto{\pgfpoint{\pgf@xb}{\pgf@ya}}
    \pgfpathclose
    % add little corner
    \pgfpathmoveto{\pgfpoint{\pgf@xc}{\pgf@yb}}
    \pgfpathlineto{\pgfpoint{\pgf@xc}{\pgf@yc}}
    \pgfpathlineto{\pgfpoint{\pgf@xb}{\pgf@yc}}
    \pgfpathlineto{\pgfpoint{\pgf@xc}{\pgf@yc}}
  }
}
\tikzset{doc/.style={draw, fill=none, shape=document}}
\tikzset{minidoc/.style={draw, fill=none, shape=minidocument}}
%}}}

%%%%%%%%%%%%%%%%%%%%%%%%%%%%%%%%%%%%%%%%%%%%%%%%%%%%%%%%%%%%%%%%%%%%%%%%%%%%%%%%
%%% MATRIX STATE %%%%%%%%%%%%%%%%%%%%%%%%%%%%%%%%%%%%%%%%%%%%%%%%%%%%%%%%%%%%%%%
%%%%%%%%%%%%%%%%%%%%%%%%%%%%%%%%%%%%%%%%%%%%%%%%%%%%%%%%%%%%%%%%%%%%%%%%%%%%%%%%
\providecommand{\State}[1]{\MatrixState{#1}}  % load library late to avoid name conflicts
\newcommand{\MatrixState}[1]{%
  \tikz[stateopts]{
    \foreach \x/\y/\s in {0/0/0, 0/1/1, 0/2/2, 0/3/3,
                          1/0/4, 1/1/5, 1/2/6, 1/3/7,
                          2/0/8, 2/1/9, 2/2/10,2/3/11,
                          3/0/12,3/1/13,3/2/14,3/3/15} {
       %
      \draw (\y+.5,-\x-.5) coordinate (s\s) coordinate (ss\x\y); % (rowwise indices)
      \draw (\x+.5,-\y-.5) coordinate (b\s); % AES indices (columnwise)
    }
    \draw (2,-2) coordinate (label);
    #1
    \draw (0,0) rectangle (4,-4);
    \foreach \x in {1,2,3} {
      \draw[raster] (0,-\x) -- ++(4,0);
      \draw[raster] (\x,0) -- ++(0,-4);
    }
  }%
}
\newcommand{\Column}[1]{%
  \tikz[stateopts]{
    \foreach \x/\y/\s in {0/0/0, 1/0/4, 2/0/8, 3/0/12} {
      \draw (\y+.5,-\x-.5) coordinate (s\s) coordinate (ss\x\y); % (rowwise indices)
    }
    #1
    \draw (0,0) rectangle (1,-4);
    \foreach \x in {1,2,3} { \draw[raster] (0,-\x) -- ++(1,0); }
  }%
}

\newcommand{\Cell}[2]{\draw (#1) node[inner sep=0pt,cellopts] {#2\vphantom{0}};}
\newcommand{\FillCell}[2][fillopts]{\fill[#1] (#2) ++(-.5,.5) rectangle +(1,-1);}
\newcommand{\Fill}[2][fillopts]{\FillCell[#1]{#2}}
\newcommand{\FillColumn}[2][fillopts]{\fill[#1] (ss0#2) ++(-.5,.5) rectangle +(1,-4);}
\newcommand{\FillDiagonal}[2][fillopts]{\foreach \dd in {0,...,3} {\pgfmathsetmacro{\dx}{mod(int(\dd)+int(#2),4)}\Fill[#1]{ss\dd\dx}}}
\newcommand{\FillAntiDiagonal}[2][fillopts]{\foreach \dd in {0,...,3} {\pgfmathsetmacro{\dx}{mod(int(#2)+4-int(\dd),4)}\Fill[#1]{ss\dd\dx}}}
\newcommand{\FillState}[1][fillopts]{\fill[#1] (0,0) rectangle (4,-4);}

% TODO check out if TiKZ matrices (tikzmanual.pdf Chapter 20) could be useful instead

%%%%%%%%%%%%%%%%%%%%%%%%%%%%%%%%%%%%%%%%%%%%%%%%%%%%%%%%%%%%%%%%%%%%%%%%%%%%%%%%
%%% ARROWS %%%%%%%%%%%%%%%%%%%%%%%%%%%%%%%%%%%%%%%%%%%%%%%%%%%%%%%%%%%%%%%%%%%%%
%%%%%%%%%%%%%%%%%%%%%%%%%%%%%%%%%%%%%%%%%%%%%%%%%%%%%%%%%%%%%%%%%%%%%%%%%%%%%%%%
\usetikzlibrary{arrows.meta}
\pgfarrowsdeclare{ipe arrow}{ipe arrow}{%{{{
  \pgfutil@tempdima=5pt\relax
  \advance\pgfutil@tempdima-1.5\pgflinewidth
  \pgfarrowsleftextend{+-\pgfutil@tempdima}%
  \pgfarrowsrightextend{+1.5\pgflinewidth}%
}{%
  \pgftransformshift{\pgfqpoint{1.5\pgflinewidth}{0pt}}%
  \pgfpathmoveto{\pgfpointorigin}%
  \pgfpathlineto{\pgfqpoint{-5pt}{-1.666pt}}%
  \pgfpathlineto{\pgfqpoint{-5pt}{+1.666pt}}%
  \pgfpathclose
  \pgfusepathqfill
}%}}}
\makeatother

%%%%%%%%%%%%%%%%%%%%%%%%%%%%%%%%%%%%%%%%%%%%%%%%%%%%%%%%%%%%%%%%%%%%%%%%%%%%%%%%
%%% STYLES %%%%%%%%%%%%%%%%%%%%%%%%%%%%%%%%%%%%%%%%%%%%%%%%%%%%%%%%%%%%%%%%%%%%%
%%%%%%%%%%%%%%%%%%%%%%%%%%%%%%%%%%%%%%%%%%%%%%%%%%%%%%%%%%%%%%%%%%%%%%%%%%%%%%%%

%%% OPERATION STYLES %%%%%%%%%%%%%%%%%%%%%%%%%%%%%%%%%%%%%%%%%%%%%%%%%%%%%%%%%%%
\tikzset{op/.style={draw, fill=none, minimum size=1.5ex, inner sep=0pt}}
\tikzset{xor/.style={op, shape=oplus}}
\tikzset{not/.style={op, shape=ominus}}
\tikzset{mul/.style={op, shape=otimes}}
\tikzset{and/.style={op, shape=otimes}} % TODO \odot
\tikzset{sum/.style={op, shape=boxplus}}
\tikzset{tee/.style={shape=circle, fill, draw, inner sep=0pt, minimum size=2pt}}
\tikzset{rot/.style={shape=rectangle, draw, inner sep=2pt, rounded corners=2pt}} % TODO

%%% PRIMITIVE STYLES %%%%%%%%%%%%%%%%%%%%%%%%%%%%%%%%%%%%%%%%%%%%%%%%%%%%%%%%%%%
\tikzset{box/.style={minimum size=.75cm, draw, inner sep=1pt, fill=white, rounded corners=3pt}}

\tikzset{
  key west/.style={path picture={\draw[line cap=round,rounded corners=0pt] (path picture bounding box.west) +(.25pt,.65ex) -- +(.65ex,0) -- +(.25pt,-.65ex);}},
  key east/.style={path picture={\draw[line cap=round,rounded corners=0pt] (path picture bounding box.east) +(-.25pt,.65ex) -- +(-.65ex,0) -- +(-.25pt,-.65ex);}},
  key north/.style={path picture={\draw[line cap=round,rounded corners=0pt] (path picture bounding box.north) +(-.65ex,-.25pt) -- +(0,-.65ex) -- +(.65ex,-.25pt);}},
  key south/.style={path picture={\draw[line cap=round,rounded corners=0pt] (path picture bounding box.south) +(-.65ex,.25pt) -- +(0,.65ex) -- +(.65ex,.25pt);}},
  key at/.style args = {#1}{path picture={\draw[line cap=round,rounded corners=0pt] (path picture bounding box.#1) +(-.65ex,.25pt) -- +(0,.65ex) -- +(.65ex,.25pt);}},
}

%%% ARROW STYLES %%%%%%%%%%%%%%%%%%%%%%%%%%%%%%%%%%%%%%%%%%%%%%%%%%%%%%%%%%%%%%%
\tikzset{next/.style={->, >=ipe arrow}}
\tikzset{prev/.style={<-, >=ipe arrow}}

%%% MATRIX STATE %%%%%%%%%%%%%%%%%%%%%%%%%%%%%%%%%%%%%%%%%%%%%%%%%%%%%%%%%%%%%%%
\tikzset{
  state/.style={inner sep=0pt},
  stateopts/.style={scale=.3},
  raster/.style={},
  fillopts/.style={gray},
  cellopts/.style={},
}

% TODO provide settings that prefill nodes with \State{}

% TODO alternative shapes -- careful, they mess up with chained comments and relative coordinates!
\tikzset{
  notalt/.style={draw,circle,append after command={
                 [shorten >=\pgflinewidth, shorten <=\pgflinewidth]
                 (\tikzlastnode.east) edge (\tikzlastnode.west)
                }},
  xoralt/.style={draw,circle,append after command={
                 [shorten >=\pgflinewidth, shorten <=\pgflinewidth]
                 (\tikzlastnode.north) edge (\tikzlastnode.south)
                 (\tikzlastnode.east) edge (\tikzlastnode.west)
                }},
  andalt/.style={op, shape=circle, every label/.style={}, label={center:\tikz{\filldraw circle (.75pt);}} },
}

% vim: foldmethod=marker
